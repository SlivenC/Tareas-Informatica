\documentclass[20pt,a4paper]{article}
\usepackage[latin1]{inputenc}
\usepackage{amsmath}
\usepackage{amsfonts}
\usepackage{amssymb}
\author{Sliven Carranza}
\begin{document}
\begin{center}
        \huge{Hoja de trabajo \#1} \\
        \large{Sliven Macario Antonio Carranza Briones 20181393} \\
\end{center} 
Ejercicio 2: \\ \\
1. Conjunto de nodos del grafo\\ \\ 
$\langle 1 , 2 , 3 , 4 , 5 , 6 \rangle$  \\\\

2. Conjunto de vertices  del grafo\\ \\
\[
Vertices= \begin{bmatrix}
\langle 1, 2 \rangle & \langle 1, 3 \rangle & \langle 1, 4 \rangle & \langle 1, 5 \rangle \\ 
\langle 2, 3 \rangle & \langle 2, 4 \rangle & \langle 2, 6 \rangle & \langle 3,5  \rangle \\ 
\langle 3, 6 \rangle & \langle 4, 5 \rangle & \langle 4, 6 \rangle & \langle 5,6  \rangle \\ 
\end{bmatrix}
 \] \\
 Ejercicio  3: \\ \\
 
 \begin{itemize}
        \item{Que estructura de datos podria representar un lanzamiento de dados?} \\ \\
       Se podria utilizar utilizar una serie de caminos por el cual se empieza con un numero  al azar que solo tendria 
        como salida 6 opciones inlcuyendo ese mismo numero. Ejemplo $\langle 1 ,2 \rangle$  $\langle 1 , 3 \rangle$  $\langle 1 , 4 \rangle$  $\langle 1 , 5 \rangle$  $\langle 1 , 6 \rangle$ $\langle 1 , 1 \rangle$ dond el camino son serie de vertices al azar hasta llegar a alguno de esos resultados.
        
        \item{Que algoritmo podriamos utilizar para generar dicha estructura?}\\ \\
        Un algoritmo que genere diferentes caminos(de entradas y salidas) con 6 posibles respuestas o destinos los cuales llevarian diferentes       combinaciones de vertices que serian generados durante el proceso  de manera aleatoria. Por ejemplo: \\
        Entra 1  y sus vertices podrian ser:
        $\langle 1 , 2 \rangle$  $\langle 2 , 4 \rangle$  $\langle 6 , 2 \rangle$  $\langle 3 , 5 \rangle$ \\
        Pero su salida 2.\\ 
        Pero se podria dar el caso donde         
        Entra 1  y sus vertices podrian ser:
        $\langle 2 ,4 \rangle$  $\langle 5 , 6 \rangle$  $\langle 1 , 2 \rangle$  $\langle 5,6  \rangle$ \\
        y de igual manera su salida 2. \\ 
        Por eso este algoritmo solo se centraria en un conjunto de 1-6 donde solo puede haber un ciclo ya que hay la posibilidad 
        de que se repita el numero de incio.
        
        \item{Como nos aseguramos que ese algoritmo siempre produce un resultado?}\\ \\
        Como el conjunto denotado es  de $\langle 1 , 2 , 3 , 4 , 5 , 6 \rangle$  sin importar el valor de inicio el resultado siempre serria uno de esos numeros asi que el algoritmo siempre tendria un resultado, se pudo haber dado la posibilidad de que pasara eso si se hubiera inlcluido el 0 en dicho conjunto pero de igual manera es un valor asi que siempre daria un valor como resultado.
        
        
\end{itemize}

\end{document}