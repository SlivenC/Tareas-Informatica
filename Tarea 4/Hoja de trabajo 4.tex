\documentclass[10pt,a4paper]{article}
\usepackage[latin1]{inputenc}
\usepackage{amsmath}
\usepackage{amsfonts}
\usepackage{amssymb}
\author{Sliven Carranza}
\begin{document}
\begin{center}
        \huge{Hoja de trabajo \#4} \\
        \large{Sliven Macario Antonio Carranza Briones 20181393} \\
\end{center}
 \section*{Ejercicio \#1 (10\%)}
A continuaci\'on se le presentara una serie de definiciones de conjuntos pertenecientes al
conjunto $2^{\mathbb{N}}$. Indicar que definiciones corresponden al mismo conjunto, es decir
que definiciones definen conjuntos que tienen los mismos elementos.
\begin{enumerate}
        \item{$a:=\{1,2,4,8,16,32,64\}$}
        \item{$b:=\{n\ \in \mathbb{N}\ |\ \exists x \in \mathbb{N}\ .\ x=n/5 \}$}
        \item{$c:=\{n\in \mathbb{N}\ |\ \exists x\in\mathbb{N}\ .\ n=x*x \}$}
        \item{$d:=\{n\in\mathbb{N}\ |\ \exists i\in\mathbb{N}\ .\ n=2^i\wedge n<100 \}$}
        \item{$e:=\{ n\in\mathbb{N}\ |\ \exists x\in \mathbb{N}\ .\ x=\sqrt{n} \}$}
        \item{$f:=\{ n\in\mathbb{N}\ |\ \exists x\in \mathbb{N}\ .\ n=x+x+x+x+x \}$} \\ \\
\end{enumerate} 
Respuestas: 
\begin{enumerate}
 \item{$(A\equiv D):\Leftrightarrow ( \forall x.x\in A\Leftrightarrow x \in D)$}
 \item{$(C\equiv F):\Leftrightarrow ( \forall x.x\in C\Leftrightarrow x \in F)$}
  \item{$(B\equiv E):\Leftrightarrow ( \forall x.x\in B\Leftrightarrow x \in E)$}
  
\end{enumerate}

\section*{Ejercicio \#2 (10\%)}
Utilize la \emph{jerga matematica} para definir los siguientes conjuntos:
\begin{enumerate}
        \item{El conjunto de todos los naturales divisibles dentro de $5$} \\
         $C:=\{n\ \in \mathbb{N}\ |\ \exists x \in \mathbb{N}\ .\ x=n/5 \}$
        \item{El conjunto de todos los naturales divisibles dentro de $4$ y $5$} \\
        $A:=\{n\ \in \mathbb{N}\ |\ \exists x \in \mathbb{N}\ .\ x=n/5 \}$\\
        $B:=\{n\ \in \mathbb{N}\ |\ \exists y \in \mathbb{N}\ .\ y=n/4\}$\\
        $A\cup B:=\{x|x\in A\vee x\in B\}$
        \item{El conjunto de todos los naturales que son primos} \\
        $B:=\{n\ \in \mathbb{N}\ |\ \exists y \in \mathbb{N}\ .\ d<= n\}$\\
        $P:=\{n\ \in \mathbb{N}\ |\ \exists x \in \mathbb{N}\ .\ [n / d = 0 ]=c , c=2 \}$\\
        $B\cup P:=\{x|x\in B\vee x\in P\}$		
        \item{El conjunto de todos los conjuntos de numeros naturales que contienen
        un numero divisible dentro de $15$}\\
       $C:=\{n\ \in \mathbb{N}\ |\ \exists d \in \mathbb{N}\ .\ d<= 15 \}$\\
        $Q:=\{n\ \in \mathbb{N}\ |\ \exists d \in \mathbb{N}\ .\ 15 / d = 0 \}$\\
        $C\cup Q:=\{x|x\in B\vee x\in P\}$		 
        \item{El conjunto de todos los conjuntos de numeros naturales que al ser sumados
        producen $42$ como resultado}\\
	   $S:=\{n\ \in \mathbb{N}\ |\ \exists r \in \mathbb{N}\ .\ r<= 42, 42/r=0 \}$\\
        $W:=\{n\ \in \mathbb{N}\ |\ \exists r \in \mathbb{N}\ .\  r0+r1+.....rn = 42 \}$\\
        $S\cup W:=\{x|x\in B\vee x\in P\}$
\end{enumerate}

\section*{Ejercicio \#3 (10\%)}
 Un numero \emph{semi-primo} es el producto de dos numeros primos. Los numeros
\emph{semiprimos} tienen la peculiaridad que nada m\'as son divisibles
entre $1$ y los dos primos de los cuales dicho numero es un producto. Un ejemplo
es el numero seis ($6=2*3$) el cual se obtiene al multiplicar los primos $2$ y $3$.
\\
     $C:=\{n\ \in \mathbb{N}\ |\ \exists y \in \mathbb{N}\ .\ d<= n\}$\\
        $B1:=\{n\ \in \mathbb{N}\ |\ \exists x \in \mathbb{N}\ .\ [n / d = 0 ]=c , c=2 \}$\\
                $B1*B2:=Semi primo$

\section*{Ejercicio \#4 (20\%)}
Utilize la \emph{jerga matematica} para definir los conjuntos a los que corresponden las
siguientes funci\'ones:
\begin{enumerate}
        \item{$f:\mathbb{N}\rightarrow\mathbb{N}$; $f(x)=x+x$} \\
                 $W:=\{n\ \in \mathbb{N}\ |\ \exists x \in \mathbb{N}\ .\ x + x  \}$\\
        \item{$g:\mathbb{N}\rightarrow\mathbb{B}$; $g(x)$ es verdadero si
        $x$ es divisible dentro de $5$, falso en caso contrario. Nota: $\mathbb{B}=
        \{\mathtt{true},\mathtt{false}\}$, puede definir dos conjuntos separados y
        definir la funci\'on como la union de ambos conjuntos.} \\
         $C:=\{n\ \in \mathbb{N}\ |\ \exists x \in \mathbb{N}\ .\ X/5=0,True \}$\\
         $B:=\{n\ \in \mathbb{N}\ |\ \exists x \in \mathbb{N}\ .\ X/5=1,False \}$\\
        $C\cup B:=\{x|x\in A\vee x\in B\}$
        \item{Indicar el conjunto al que pertenece la funci\'on $f\circ g$}\\
         $C:=\{n\ \in \mathbb{N}\ |\ \exists x \in \mathbb{N}\ .\in\ \}$\\
        \item{Definir el conjunto que corresponde a la funci\'on $f\circ g$}\\
        $C:=\{n\ \in \mathbb{N}\ |\ \exists x \in \mathbb{N}\ .\times\mathbb{Z} \}$\\

\end{enumerate}

\section*{Ejercicio \#5 (20\%)}
Dadas las siguientes funciones que pertenecen a $\mathbb{R}\rightarrow \mathbb{R}$, indique
si la funci\'on es injectiva, surjectiva o bijectiva.
\begin{enumerate}
        \item{$f(x)=x^2$} \\
        Surjectiva \\ 
        \item{$g(x)=\frac{1}{cos(x-1)}$} \\
        Surjectiva\\
        \item{$h(x)=2x$} \\
        Bijectiva\\
        \item{$w(x)=x+1$}\\
        Bijectiva\\
\end{enumerate}
\section*{Ejercicio \#6 (30\%)}
A continuaci\'on se definira una bijecci\'on entre los numeros naturales ($\mathbb{N}$) y los
numeros enteros ($\mathbb{Z}$). Se utilizaran varios conjuntos intermediariarios para facilitar
el proceso.
\begin{enumerate}
        \item{Definir el conjunto $B_1\in \mathbb{N}\times\mathbb{N}$ el cula empareja a los
        numeros naturales \emph{pares} con todos los naturales mayores a $0$. Eg. $B_1=\{
        \langle 2,1 \rangle, \langle 4,2 \rangle, \langle 6, 3 \rangle\ldots \}$} \\
        $b:=\{n\ \in \mathbb{N}\ |\ \exists x \in \mathbb{N}\ .\ x/2=0 \}$\\
        \item{Definir el conjunto $B_{2a}\in \mathbb{N}\times\mathbb{N}$ el cula empareja a los
        numeros naturales \emph{impares} con todos los naturales mayores a $0$. Eg. $B_{2a}=\{
        \langle 1,1 \rangle, \langle 3,2 \rangle, \langle 5, 3 \rangle\ldots \}$} \\
         $c:=\{n\ \in \mathbb{N}\ |\ \exists x \in \mathbb{N}\ .\ x/2=1 \}$\\
        \item{Definir el conjunto $B_{2}\in \mathbb{N}\times\mathbb{Z}$ el cual se definie
        exactamente igual al conjunto $B_{2a}$ excepto que los valores en el contradominio
        son negativos}
         $d:=\{n\ \in \mathbb{N}\ |\ \exists x \in \mathbb{N}\ .\ n * z \}$\\
        \item{El conjutno $B:= \{\langle 0,0\rangle \}\cup B_{1} \cup B_{2}$ es la bijeccion
        que se intenta definir.}
\end{enumerate}

\end{document}