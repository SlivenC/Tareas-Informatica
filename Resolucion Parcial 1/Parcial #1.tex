\documentclass[10pt,a4paper]{article}
\usepackage[latin1]{inputenc}
\usepackage{amsmath}
\usepackage{amsfonts}
\usepackage{amssymb}
\author{Sliven Carranza}
\begin{document}
\begin{center}
        \huge{Examen Parcia Resolucion \#1} \\
        \large{Sliven Macario Antonio Carranza Briones} \\
        \large{20181393} \\
\end{center}
\section{Pregunta \#1 (20\%)}
El famoso matematico Euler hizo la siguiente pregunta: Es posible curzar todos los puentes
de K\"onigsberg sin pasar dos veces por el mismo puente? A continuaci\'on se muestra un
mapa de los puentes de K\"onigsberg:
\begin{itemize}
        \item{Definir el conjunto de nodos} \\\\
        $\langle 1 , 2 , 3 , 4 , 5 , 6 ,7 \rangle$  \\\\
        \item{Definir el conjunto de vertices}      
\end{itemize}
\[
 Vertices= \begin {bmatrix}
\langle 1, 5 \rangle & \langle 5, 6 \rangle & \langle 1, 2 \rangle & \langle 1, 4 \rangle  & \langle 1, 3 \rangle \\ 
\langle 1, 6 \rangle & \langle 2, 3 \rangle & \langle 2, 5 \rangle & \langle 2, 4 \rangle  & \langle 2, 6 \rangle \\
\langle 3, 6 \rangle & \langle 3, 4 \rangle & \langle 3, 7 \rangle & \langle 5, 7 \rangle  & \langle 6, 7 \rangle \\
\langle 7, 4 \rangle & \langle  3, 5 \rangle 
\end {bmatrix}
\] \\
\section*{Pregunta \#2 (20\%)}
Demostrar utilizando inducci\'on que la formula de Gauss para sumatorias es correcta:
\[
        \sum_{i=1}^{n}{i}=\frac{n(n+1)}{2}
\]
donde $\sum_{i=1}^{n}i=1+2+3+4+\ \ldots\ +n$.
\\\\
Para esta demostraci\'on, su caso base debe ser
$n=1$ en vez de $n=0$. Sin embargo, la demostraci\'on
del caso inductivo procede de la misma forma que
se ha estudiado en clase.
\\ \\ \\ \\ \\ \\ \\ \\ \\ \\ \\ \\ \\ \\
Demostracion:
\begin{itemize}
        \item{Caso Base}
        n=1 
            \[ \sum_{i=1}^{n}{i}=\frac{n(n+1)}{2} \] \\
            \[1 =\frac{1(1+1)}{2}  \]\\
            \[ 1 =\frac{1(2)}{2} \] \\
              \[ 1 =\frac{(2)}{2} \] \\
              \[ 1 = 1  \] \\
        \item{Caso inductivo n+1}
            \[ \sum_{i=1}^{n+1}{i}=\frac{(n+1)(n+1+1)}{2} \] \\
            \[ \sum_{i=1}^{n}{i}+(n+1)=\frac{(n+1)(n+2)}{2} \] \\
Esto es igual a:
            \[\frac{n(n+1)}{2} + (n+1)  =\frac{(n+1)(n+2)}{2} \] \\
            \[\frac{n(n+1)+2(n+1)} {2}   =\frac{(n+1)(n+2)}{2} \] \\
            \[\frac{n^2 +n +2n+2} {2}   =\frac{n^2 +2n +n+2}{2} \] \\
           \[\frac{n^2 +3n+2} {2}   =\frac{n^2 +3n+2}{2} \] \\
\end{itemize}
\section*{Pregunta \#3 (20\%)}
Definir inductivamente la funcion $\sum(n)$ para numeros naturales unarios la cual tiene
el efecto de calcular la suma de $1$ hasta $n$. En otras palabras:
\[
        \sum(n)=1+2+3+4+\ \ldots\ +n
\]
Puede apoyarse de la suma $\oplus$ de numeros naturales unarios para su definici\'on:
\[
        a\oplus b =
                \left\{
                        \begin{array}{ll}
                                b  & \mbox{si } a = 0 \\
                                s(i\oplus b) & \mbox{si } a = s(i)
                        \end{array}
                \right.
\]
Demostracion:  i= 1 , n=2 \\
        \[ \sum(n)=(s(i) + i) \]
        \[ 2  =  (1+1)  \]
        \[ 2  =  (2)  \]
    Respuesta:  \[ \sum(n)=(s(i) + i) \]
    
\section*{Pregutna \# 4 (20\%)}
Demostrar por medio de inducci\'on la comutatividad de la suma de
numeros naturales unarios: $a\oplus b = b\oplus a$ \\ \\
Demostracion\\ \\
a = s(0)\\
b =s( s(0))\\
$a\oplus b = b\oplus a$ \\
$s(0) \oplus s( s(0)) = s(s(0)) \oplus  s(0), $ \\
$s(s(0)) \oplus  s(0) = s(ss(0)) \oplus  0, $\\
$s(ss(0)) \oplus  0 = s(ss(0))  $\\
$s(ss(0)) = s(ss(0))  $\\

\section*{Pregunta \#5 (20\%)}
Dada la funci\'on $a\geq b$ para numeros naturales unarios:
\[
        a\geq b =
                \left\{
                        \begin{array}{ll}
                                s(o)  & \mbox{si } b = o \\
                                o & \mbox{si } a = o \\
                                i\geq j & \mbox{si } a = s(i)\ \&\ b = s(j)
                        \end{array}
                \right.
\]
Demostrar utilizando inducci\'on que $((n\oplus n)\geq n) = s(o)$. Puede
hacer uso de la asociatividad y comutabilidad de la suma de numeros
unarios para su demostraci\'on.  \\\\
Demostracion\\
Caso Base: 
n=0\\ 
$((n\oplus n)\geq n) = s(o)$ \\
$((0\oplus 0)\geq 0) = 0$ \\
$((0\oplus 0)\geq 0) = 0$ \\
$(0) = 0$ \\

Caso Inductivo \\
$((n\oplus n)\geq n) = s(0)$ \\
$((i \oplus j )\geq s(j)) = s(0)$ \\
$((n\oplus n)\geq s(i)) = s(0)$ \\
$(0)\geq s(i)) = s(0)$ \\
$(0)\geq s(i)) = s(0)$ \\
Dado las desigualdades de los numeros unarios:   
$ s(0) = s(0)$ \\

\end{document}